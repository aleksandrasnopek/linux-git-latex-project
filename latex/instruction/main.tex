\documentclass[a4paper,12pt]{article}

\usepackage[utf8]{inputenc}
\usepackage[T1]{fontenc}
\usepackage[polish]{babel}
\usepackage{geometry}
\usepackage{graphicx}
\usepackage{hyperref}

\geometry{margin=2.5cm}
\title{Instrukcja obsługi GitHub Desktop}
\author{Student}
\date{\today}

\begin{document}

\maketitle
\tableofcontents
\newpage

\section{Wprowadzenie}
Niniejsza instrukcja opisuje szczegółowo obsługę aplikacji GitHub Desktop.
Przedstawia proces logowania, klonowania repozytorium, wprowadzania zmian,
tworzenia commitów, synchronizacji z repozytorium zdalnym oraz tworzenia Pull Request.

\section{Logowanie do GitHub Desktop}
Po uruchomieniu aplikacji GitHub Desktop użytkownik musi zalogować się
na swoje konto GitHub.

\begin{figure}[h]
\centering
\includegraphics[width=\linewidth]{screenshots/login.png}
\caption{Logowanie do aplikacji GitHub Desktop}
\end{figure}

\section{Klonowanie repozytorium}
Po zalogowaniu należy wybrać opcję \textit{Clone a repository},
a następnie wskazać repozytorium oraz lokalizację na dysku.

\begin{figure}[h]
\centering
\includegraphics[width=\linewidth]{screenshots/clone2.png}
\caption{Wybór repozytorium do klonowania}
\end{figure}

\begin{figure}[h]
\centering
\includegraphics[width=\linewidth]{screenshots/repo_view.png}
\caption{Widok repozytorium po klonowaniu}
\end{figure}

\section{Wprowadzanie zmian w projekcie}
Po otwarciu repozytorium użytkownik może edytować pliki projektu.
GitHub Desktop automatycznie wykrywa wszystkie zmiany.

\begin{figure}[h]
\centering
\includegraphics[width=\linewidth]{screenshots/changes2.png}
\caption{Wykryte zmiany w plikach projektu}
\end{figure}

\section{Tworzenie commitów}
Commit zapisuje aktualny stan projektu wraz z opisem wprowadzonych zmian.
Aby go utworzyć, należy wpisać krótki komunikat i zatwierdzić zmiany.

\begin{figure}[h]
\centering
\includegraphics[width=\linewidth]{screenshots/commit.png}
\caption{Tworzenie commita}
\end{figure}

\begin{figure}[h]
\centering
\includegraphics[width=\linewidth]{screenshots/history.png}
\caption{Historia commitów w GitHub Desktop}
\end{figure}

\section{Wysyłanie zmian do repozytorium zdalnego}
Po wykonaniu commitów zmiany znajdują się lokalnie.
Opcja \textit{Push origin} umożliwia ich wysłanie do repozytorium zdalnego.

\begin{figure}[h]
\centering
\includegraphics[width=\linewidth]{screenshots/push.png}
\caption{Wysyłanie zmian do GitHuba}
\end{figure}

\section{Tworzenie Pull Request}
Pull Request służy do połączenia zmian z gałęzi roboczej
z gałęzią główną repozytorium.

\begin{figure}[h]
\centering
\includegraphics[width=\linewidth]{screenshots/pullrequest.png}
\caption{Tworzenie Pull Request}
\end{figure}

\section{Podsumowanie}
GitHub Desktop umożliwia intuicyjną pracę z systemem kontroli wersji Git.
Dzięki przejrzystemu interfejsowi użytkownik może efektywnie zarządzać
repozytoriami bez konieczności korzystania z linii poleceń.

\end{document}
