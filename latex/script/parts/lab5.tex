\section{Instrukcje Warunkowe oraz Pętle}
Prawdziwa moc skryptów polega na ich zdolności do \textbf{powtarzania} operacji i \textbf{podejmowania decyzji}. Te dwie koncepcje -- pętle i instrukcje warunkowe -- zamieniają prostą listę poleceń w inteligentny program.

\subsection{Instrukcje warunkowe: \texttt{if}}
Instrukcja \texttt{if} pozwala skryptowi na wykonanie określonego bloku kodu tylko wtedy, gdy pewien warunek jest prawdziwy. Działa to jak rozwidlenie dróg -- skrypt wybiera ścieżkę w zależności od sytuacji.

\subsection*{Anatomia instrukcji \texttt{if}}
Strukturę można rozbudowywać o kolejne warunki (\texttt{elif}) oraz blok domyślny (\texttt{else}), który wykona się, gdy żaden z poprzednich warunków nie będzie prawdziwy.
\begin{verbatim}
if [[ warunek_1 ]]; then
    # Ten kod wykona się, jeśli warunek_1 jest prawdziwy.
    # Bash przejdzie od razu do `fi`.
elif [[ warunek_2 ]]; then
    # Ten kod wykona się, jeśli warunek_1 był fałszywy,
    # ale warunek_2 jest prawdziwy.
else
    # Ten kod wykona się, jeśli wszystkie powyższe warunki
    # okazały się fałszywe.
fi
\end{verbatim}
\textbf{Kluczowa składnia:} W Bashu warunki najczęściej umieszcza się w podwójnych nawiasach kwadratowych \texttt{[[ ... ]]}. Jest to nowoczesna i bezpieczna forma.


\subsection*{Operatory testów – Twój zestaw narzędzi do sprawdzania warunków}
Bash oferuje bogaty zestaw operatorów do budowania warunków. Oto najważniejsze z nich:
\begin{itemize}
    \item \textbf{Testy plików (najczęstsze w skryptach):}
    \begin{itemize}
        \item \texttt{[[ -e \$sciezka ]]} – sprawdza, czy plik lub katalog \textbf{e}gzistuje (\textit{exists}).
        \item \texttt{[[ -f \$sciezka ]]} – sprawdza, czy to zwykły \textbf{f}ile (plik).
        \item \texttt{[[ -d \$sciezka ]]} – sprawdza, czy to \textbf{d}irectory (katalog).
    \end{itemize}
    \item \textbf{Testy napisów (zmiennych):}
    \begin{itemize}
        \item \verb|[[ -z "\$zmienna" ]]| – prawda, jeśli zmienna jest pusta (\textbf{z}ero length).
        \item \verb|[[ -n "\$zmienna" ]]| – prawda, jeśli zmienna \textbf{n}ie jest pusta (\textit{non-empty}).
        \item \verb|[[ "\$a" == "\$b" ]]| – prawda, jeśli napisy są identyczne.
    \end{itemize}
    \item \textbf{Testy liczb całkowitych:}
    \begin{itemize}
        \item \verb|[[ \$a -eq \$b ]]| – równe (\textit{equal}).
        \item \texttt{[[ \$a -ne \$b ]]} – nie równe (\textit{not equal}).
        \item \texttt{[[ \$a -gt \$b ]]} – większe (\textit{greater than}).
        \item \texttt{[[ \$a -lt \$b ]]} – mniejsze (\textit{less than}).
    \end{itemize}
    \textbf{Operator negacji \texttt{!}:}
    \begin{itemize}
        \item \verb|[[ ! -d "\$sciezka" ]]| – prawda, jeśli Sciezka nie jest katalogiem.
    \end{itemize}
\end{itemize}



\subsection{Pętle: \texttt{for} i \texttt{while}}
Pętle służą do wielokrotnego wykonywania tego samego bloku kodu. Bez nich musielibyśmy ręcznie powielać polecenia dla każdego przetwarzanego elementu.

\subsubsection{Pętla \texttt{for} -- iteracja po liście elementów}
Pętla \texttt{for} jest idealna, gdy mamy z góry znaną listę elementów (np. listę plików w katalogu) i chcemy wykonać jakąś operację dla każdego z nich.

\subsection*{Anatomia pętli \texttt{for}}
\begin{verbatim}
for zmienna in element1 element2 element3; do
    # Ten blok kodu wykona się 3 razy.
    # W pierwszej iteracji $zmienna będzie miała wartość "element1",
    # w drugiej "element2", itd.
    echo "Przetwarzam: $zmienna"
done
\end{verbatim}
Najczęstszym zastosowaniem jest iteracja po plikach: \texttt{for plik in sciezka/*}. Powłoka Bash automatycznie rozwija wzorzec \texttt{*} na listę wszystkich plików i katalogów w danej lokalizacji.



\subsubsection{Pętla \texttt{while} -- wykonywanie do spełnienia warunku}
Pętla \texttt{while} działa inaczej: wykonuje blok kodu tak długo, jak długo jej warunek jest prawdziwy. Jest idealna w sytuacjach, gdy nie wiemy, ile będzie iteracji, np. podczas wczytywania danych z pliku linia po linii.

\subsection*{Anatomia pętli \texttt{while}}
\begin{verbatim}
# Przykład: odliczanie od 5 do 1
licznik=5
while [[ $licznik -gt 0 ]]; do
    echo "Odliczanie: $licznik"
    # Ważne: musimy samodzielnie modyfikować warunek,
    # w przeciwnym razie pętla będzie nieskończona!
    licznik=$((licznik - 1))
done
echo "Start!"
\end{verbatim}
Pętla \texttt{while} sprawdza warunek na początku każdej iteracji. Jeśli warunek od razu jest fałszywy, pętla nie wykona się ani razu.


\section{Przekazywanie argumentów do skryptu}
Skrypty mogą przyjmować argumenty z wiersza poleceń. Wewnątrz skryptu mamy dostęp do tych argumentów za pomocą specjalnych zmiennych:
\begin{itemize}
    \item \texttt{\$0} – nazwa samego skryptu.
    \item \texttt{\$1}, \texttt{\$2}, ... – pierwszy, drugi, itd. argument.
    \item \texttt{\$\#} – całkowita liczba przekazanych argumentów.
\end{itemize}

\section{Przetwarzanie Tekstu i Potoki}
Jedną z najpotężniejszych cech powłoki jest możliwość łączenia prostych narzędzi w złożone sekwencje za pomocą \textbf{potoków (pipes)}, reprezentowanych przez znak \texttt{|}.

\subsection*{Potok (pipe)}
Potok przekierowuje standardowe wyjście jednego polecenia na standardowe wejście następnego. Działa to jak linia montażowa: każde narzędzie wykonuje swoją pracę, a wynik przekazuje dalej.


W skryptach laboratoryjnych używamy potoków do wyodrębnienia kategorii z pliku. Przeanalizujmy tę linię:
\begin{verbatim}
category=$(grep -i '^CATEGORY:' "$file" | cut -d':' -f2 | tr -d '[:space:]')
\end{verbatim}
\begin{itemize}
    \item \texttt{grep -i '\textasciicircum CATEGORY:' "\$file"} \\
    \textit{Etap 1: Filtrowanie.} Polecenie \texttt{grep} odnajduje w pliku interesującą nas linię.
    \begin{itemize}
        \item \texttt{-i}: Ignoruj wielkość liter (\textit{ignore case}).
        \item \texttt{'\textasciicircum CATEGORY:'}: Wzorzec. Znak \texttt{\textasciicircum} to "kotwica", która oznacza "początek linii". Szukamy więc linii, które \textbf{zaczynają się} od \texttt{CATEGORY:}.
    \end{itemize}
    
    \item \texttt{cut -d':' -f2} \\
    \textit{Etap 2: Wycinanie.} Wynik z \texttt{grep} (cała linia) jest przekazywany do \texttt{cut}, które wycina z niego tylko potrzebny fragment.
    \begin{itemize}
        \item \texttt{-d':'}: Ustawia separator (delimiter) na dwukropek. Dzieli linię na pola względem \texttt{:}.
        \item \texttt{-f2}: Każe wybrać drugie pole (\textit{field}), czyli wszystko po pierwszym dwukropku.
    \end{itemize}
    
    \item \texttt{tr -d '[:space:]'} \\
    \textit{Etap 3: Czyszczenie.} Fragment tekstu z \texttt{cut} jest przekazywany do \texttt{tr}, które usuwa z niego niepotrzebne białe znaki.
    \begin{itemize}
        \item \texttt{-d}: Usuń znaki (\textit{delete}).
        \item \texttt{'[:space:]'}: Specjalna klasa znaków oznaczająca wszystkie białe znaki (spacje, tabulatory, itp.).
    \end{itemize}
\end{itemize}

\section{Tablice Asocjacyjne}
Gdy potrzebujemy przechowywać zbiór powiązanych danych (np. mapowanie sum kontrolnych na ścieżki plików), idealnym rozwiązaniem są tablice asocjacyjne.

\subsection*{Tablica Asocjacyjna}
To struktura danych przechowująca pary \textbf{klucz-wartość}, podobnie do słownika w Pythonie. Deklarujemy ją za pomocą \texttt{declare -A nazwa\_mapy}.


\subsection*{Sprawdzanie istnienia klucza}
Niezawodnym sposobem na sprawdzenie, czy klucz \textbf{istnieje} w tablicy, jest składnia: \texttt{[[ -n "\$\{mapa[\$klucz]+x\}" ]]}


\subsection*{Tablice asocjacyjne w praktyce: śledzenie duplikatów}
Wyobraźmy sobie, jak skrypt do wyszukiwania duplikatów używa tablicy \texttt{checksum\_map} do śledzenia napotkanych plików:
\begin{enumerate}
    \item Skrypt analizuje plik \texttt{raport.txt}. Oblicza jego sumę kontrolną: \texttt{abc...}.
    \item Sprawdza, czy klucz \texttt{abc...} istnieje w mapie. \textbf{Nie istnieje}.
    \item Dodaje wpis do mapy: \texttt{checksum\_map[abc...]} = \texttt{"/home/user/raport.txt"}.
    \item Skrypt analizuje plik \texttt{dane.csv}. Oblicza jego sumę: \texttt{def...}.
    \item Sprawdza, czy klucz \texttt{def...} istnieje w mapie. \textbf{Nie istnieje}.
    \item Dodaje wpis: \texttt{checksum\_map[def...]} = \texttt{"/home/user/dane.csv"}.
    \item Skrypt analizuje plik \texttt{raport\_kopia.txt}, który jest identyczny jak pierwszy.
    \item Oblicza jego sumę kontrolną: \texttt{abc...}.
    \item Sprawdza, czy klucz \texttt{abc...} istnieje w mapie. \textbf{Tak, istnieje!}
    \item Skrypt wie, że plik \texttt{raport\_kopia.txt} jest duplikatem, ponieważ jego "odcisk palca" jest już w kartotece. Raportuje znalezisko.
\end{enumerate}


\section{Przetwarzanie Plików: \texttt{find} i \texttt{while read}}

\subsection{Problem z "trudnymi" nazwami plików}
Pętla \texttt{for plik in *} zawodzi, gdy nazwy plików zawierają spacje lub inne znaki specjalne.

\subsection{Rozwiązanie: Połączenie \texttt{find} i \texttt{while read}}
Standardowym i w 100\% niezawodnym sposobem na przetwarzanie listy plików jest połączenie polecenia \texttt{find} z pętlą \texttt{while read}.

\subsection*{Wzorzec niezawodnego przetwarzania plików}
\begin{verbatim}
while IFS= read -r -d '' nazwa_pliku; do
    # Tutaj bezpiecznie przetwarzamy plik, którego ścieżka jest w "$nazwa_pliku"
    echo "Przetwarzam: $nazwa_pliku"
done < <(find /sciezka/startowa -type f -print0)
\end{verbatim}
\begin{itemize}
    \item \texttt{find ... -print0}: Generuje strumień ścieżek oddzielonych znakiem \textbf{NULL}.
    \item \texttt{< <(...)}: \textbf{Podstawienie procesu} przekierowuje ten strumień na wejście pętli \texttt{while}.
    \item \texttt{IFS= read -r -d ''}: Każe poleceniu \texttt{read} czytać dane aż do napotkania znaku NULL (\texttt{-d ''}), bez interpretowania znaków specjalnych (\texttt{-r}) i bez dzielenia według spacji (\texttt{IFS=}).
\end{itemize}


\section{Sumy Kontrolne}
W skryptach do wyszukiwania duplikatów kluczową rolę odgrywa mechanizm sum kontrolnych. Pozwala on w sposób jednoznaczny i efektywny sprawdzić, czy dwa pliki są identyczne pod względem zawartości, ignorując ich nazwy czy daty modyfikacji.

\subsection*{Suma Kontrolna (Checksum)}
Suma kontrolna to \textbf{cyfrowy odcisk palca} pliku. Jest to unikalny, krótki ciąg znaków o stałej długości, wygenerowany na podstawie całej zawartości pliku za pomocą algorytmu kryptograficznego.


Właściwości sumy kontrolnej są analogiczne do ludzkiego odcisku palca:
\begin{itemize}
    \item \textbf{Unikalność:} Dwa różne pliki prawie na pewno będą miały zupełnie inne sumy kontrolne.
    \item \textbf{Determinizm:} Ten sam plik zawsze wygeneruje dokładnie tę samą sumę kontrolną.
    \item \textbf{Efekt lawinowy:} Nawet najmniejsza zmiana w pliku (np. zmiana jednej litery) powoduje, że nowa suma kontrolna jest \textbf{całkowicie inna}.
    \item \textbf{Jednokierunkowość:} Na podstawie samej sumy kontrolnej nie da się odtworzyć oryginalnego pliku.
\end{itemize}

\subsection*{\texttt{sha256sum} – Cyfrowy Skaner Odcisków Palców}
\texttt{sha256sum} to standardowe polecenie w Linuksie, które implementuje algorytm \textbf{SHA-256} (Secure Hash Algorithm, 256-bit). Jest to "urządzenie", które pobiera cyfrowy odcisk palca. Wynikiem jego działania jest 256-bitowy hash, reprezentowany jako 64 znaki szesnastkowe.


\subsection*{Efekt lawinowy w praktyce}
Zobacz, jak dramatycznie zmienia się suma kontrolna po dodaniu jednego znaku:
\begin{verbatim}
# Używamy `echo -n`, aby uniknąć dodania znaku nowej linii na końcu
$ echo -n "Witaj swiecie" | sha256sum
2db652244951239987819875e3f3ac7c9615a1334237c1a8397a216439e6a037  -

$ echo -n "Witaj swiecie." | sha256sum
266228333f28d81084b423f7e5015e58983e20042472d423982e5d52e5b74108  -
\end{verbatim}


\newpage
\section{Laboratorium: Analiza i Modyfikacja Skryptów}
Celem laboratorium jest zrozumienie działania, a następnie samodzielne rozbudowanie trzech skryptów automatyzujących pracę z plikami.

\subsection*{Skrypt 1: Automatyczne Sortowanie Plików}
\textbf{Przeznaczenie:} Skrypt automatycznie klasyfikuje pliki z katalogu \texttt{inbox} do podkatalogów w folderze \texttt{classified} na podstawie zawartości każdego pliku.
\begin{verbatim}
#!/usr/bin/env bash
INCOMING="inbox"; OUTDIR="classified"
mkdir -p "$OUTDIR"
for file in "$INCOMING"/*; do
    [ -e "$file" ] || continue
    category=$(grep -i '^CATEGORY:' "$file" | cut -d':' -f2 | tr -d '[:space:]')
    if [[ -z "$category" ]]; then
        category="unknown"
    fi
    mkdir -p "$OUTDIR/$category"
    cp "$file" "$OUTDIR/$category"/
    echo "Copied $(basename "$file") -> $OUTDIR/$category/"
done
echo "Classification complete."
\end{verbatim}
\textbf{Kluczowe koncepcje:} Pętla \texttt{for}, potoki (`|`), polecenia `grep`, `cut`, `tr`, instrukcja warunkowa \texttt{if}.


\subsection{Skrypt 2: Wyszukiwanie Duplikatów (Wersja Podstawowa)}
\textbf{Przeznaczenie:} Skrypt znajduje duplikaty plików w jednym, wskazanym katalogu. Porównuje pliki na podstawie ich zawartości za pomocą sum kontrolnych \texttt{sha256sum}.
\begin{verbatim}
#!/usr/bin/env bash
if [[ $# -ne 1 ]]; then echo "Usage: $0 <directory>"; exit 1; fi
DIR="$1"
if [[ ! -d "$DIR" ]]; then echo "Error: '$DIR' is not a directory."; exit 1; fi
declare -A checksum_map
for file in "$DIR"/*; do
    [[ -f "$file" ]] || continue
    checksum=$(sha256sum "$file" | cut -d ' ' -f1)
    if [[ -n "${checksum_map[$checksum]+x}" ]]; then
        echo "Duplicate found:"
        echo "  Original: ${checksum_map[$checksum]}"; echo "  Duplicate: $file"
    else
        checksum_map[$checksum]="$file"
    fi
done
\end{verbatim}
\textbf{Kluczowe koncepcje:} Argumenty skryptu, walidacja danych, tablice asocjacyjne, sumy kontrolne.


\subsection*{Skrypt 3: Wyszukiwanie Duplikatów (Wersja Zaawansowana)}
\textbf{Przeznaczenie:} Ulepszona wersja poprzedniego skryptu. Jest rekurencyjna i bezpieczna dla nietypowych nazw plików.
\begin{verbatim}
#!/usr/bin/env bash
if [[ $# -ne 1 || ! -d "$1" ]]; then echo "Usage: $0 <directory>"; exit 1; fi
DIR="$1"
declare -A checksum_map
while IFS= read -r -d '' file; do
    checksum=$(sha256sum "$file" | cut -d ' ' -f1)
    if [[ -n "${checksum_map[$checksum]+x}" ]]; then
        echo "Duplicate found:"
        echo "  Original: ${checksum_map[$checksum]}"; echo "  Duplicate: $file"
    else
        checksum_map[$checksum]="$file"
    fi
done < <(find "$DIR" -type f -print0)
\end{verbatim}
\textbf{Kluczowe koncepcje:} Niezawodne przetwarzanie plików za pomocą \texttt{find -print0} i pętli \texttt{while read}.



\subsection*{Zadania do wykonania}
\begin{enumerate}
    \item \textbf{Modyfikacja Skryptu 1 (Sortowanie):}
    \begin{itemize}
        \item Zmień skrypt tak, aby \textbf{przenosił} pliki (\texttt{mv}) zamiast je kopiować (\texttt{cp}).
        \item Dodaj obsługę drugiego argumentu, który będzie określał katalog docelowy (zamiast na stałe wpisanego \texttt{"classified"}). Jeśli argument nie zostanie podany, skrypt powinien użyć wartości domyślnej.
    \end{itemize}
    \item \textbf{Modyfikacja Skryptu 3 (Duplikaty):}
    \begin{itemize}
        \item Dodaj do skryptu nową funkcjonalność: po znalezieniu duplikatu, skrypt powinien zapytać użytkownika, czy chce go usunąć. Użyj polecenia \texttt{read -p "Pytanie" zmienna}, aby wczytać odpowiedź. Wewnątrz pętli `while` dodaj instrukcję warunkową, która sprawdzi odpowiedź użytkownika (np. "t" lub "T").
        \item (Zaawansowane) Rozbuduj skrypt tak, aby raportował nie tylko pierwszy duplikat, ale wszystkie. (Wskazówka: wartość w tablicy asocjacyjnej może być listą plików, a nie pojedynczym plikiem).
    \end{itemize}
\end{enumerate}
